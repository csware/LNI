% !TeX encoding = UTF-8
% !TeX spellcheck = de_DE

%% Dies gibt Warnungen aus, sollten veraltete LaTeX-Befehle verwendet werden
\RequirePackage[l2tabu, orthodox]{nag}

\documentclass[utf8,biblatex]{lni}
\bibliography{lni-paper-example-de}

%% Schöne Tabellen mittels \toprule, \midrule, \bottomrule
\usepackage{booktabs}

%% Zu Demonstrationszwecken
\usepackage[math]{blindtext}
\usepackage{mwe}

%% BibLaTeX-Sonderkonfiguration,
%% falls man schnell eine existierende Bibliographie wiederverwenden will, aber nicht die .bib-Datei händisch anpassen möchte.
%% Bitte \iffalse und \fi entfernen, dann ist diese Konfiguration aktiviert.

\iffalse
\AtEveryBibitem{%
  \ifentrytype{article}{%
  }{%
    \clearfield{doi}%
    \clearfield{issn}%
    \clearfield{url}%
    \clearfield{urldate}%
  }%
  \ifentrytype{inproceedings}{%
  }{%
    \clearfield{doi}%
    \clearfield{issn}%
    \clearfield{url}%
    \clearfield{urldate}%
  }%
}
\fi

\begin{document}
%%% Mehrere Autoren werden durch \and voneinander getrennt.
%%% Die Fußnote enthält die Adresse sowie eine E-Mail-Adresse.
%%% Das optionale Argument (sofern angegeben) wird für die Kopfzeile verwendet.
\title[Ein Kurztitel]{Der Titel des Beitrags, der auch etwas länger sein und über zwei Zeilen reichen kann}
%%%\subtitle{Untertitel / Subtitle} % falls benötigt
\author[Vorname1 Nachname1 \and Vorname2 Nachname2]
{Vorname1 Nachname1\footnote{Universität, Abteilung, Straße, Postleitzahl Ort, Land \email{emailaddress@author1}} \and
 Vorname2 Nachname2\footnote{University, Department, Address, Country \email{emailaddress@author2}}}
\startpage{11} % Beginn der Seitenzählung für diesen Beitrag
\editor{Herausgeber et al.}    % Namen der Herausgeber
\booktitle{Name-der-Konferenz} % Name des Tagungsband; optional Kurztitel
\yearofpublication{2017}
%%%\lnidoi{18.18420/provided-by-editor-02} % Falls bekannt
\maketitle

\begin{abstract}
Es folgt ein kurzer Überblick über die Arbeit, der zwischen 70 und 150 Wörtern lang sein und das Wichtigste enthalten sollte. Die Schriftart ist Times New Roman in der Schriftgröße 9. Der Absatz ist im Blocksatz formatiert und beginnt immer mit den Wort „Abstract:“ ebenfalls in Schriftgröße 9, Times New Roman und fett. Vor dem Absatz sollte ein Abstand von 30 pt, danach von 6 pt und ein einfacher Zeilenabstand eingestellt werden. Es ist die ganze Seitenbreite auszunutzen. fucking shit
\end{abstract}

\begin{keywords}
Hier stehen Stichworten, die das Thema des Beitrags am besten beschreiben. Die Formatierung ist äquivalent zu der des Abstracts, nur ist vor dem Absatz ein Abstand von 6pt statt 30pt.
\end{keywords}

\section{Einleitung}
Die vorliegenden Richtlinien (Stand der Richtlinien: September 2017) sind bindend für die Erstellung von reproduktionsfertigen Manuskripten der GI-Edition ‚LNI‘ (GI = Gesellschaft für Informatik e. V.; LNI = Lecture Notes in Informatics). Sie dienen der Sicherstellung eines einheitlichen und guten Erscheinungsbildes der Schriftenreihe. Gleichzeitig kann dieses Dokument als Muster herangezogen werden, da es den Richtlinien entsprechend formatiert ist. Zudem enthält es Hinweise zur barrierefreien Aufbereitung des Dokumentes.

Die Schriftenreihe LNI umfasst Tagungsbände, die Seminarreihe der Informatiktage und die im Rahmen des Dissertationspreises der GI prämierten Dissertationen. Soweit hier Unterschiede in der Formatierung bestehen, wird im Folgenden explizit darauf hingewiesen. Legen Sie für Worte, Wortgruppen, Absätze oder Abschnitte, die von der Dokumentsprache abweichen (wie beispielsweise englische Fachbegriffe in einem deutschen Text), die entsprechende Sprache fest, damit der Screenreader ebenfalls die Intonation wechselt (Pfad: Überprüfen > Sprache > Sprache für die Korrekturhilfen festlegen). Legen Sie für Worte, Wortgruppen, Absätze oder Abschnitte, die von der Dokumentsprache abweichen (wie beispielsweise englische Fachbegriffe in einem deutschen Text), die entsprechende Sprache fest, damit der Screenreader ebenfalls die Intonation wechselt (Pfad: Überprüfen > Sprache > Sprache für die Korrekturhilfen festlegen). Legen Sie für Worte, Wortgruppen, Absätze oder Abschnitte, die von der Dokumentsprache abweichen (wie beispielsweise englische Fachbegriffe in einem deutschen Text), die entsprechende Sprache fest, damit der Screenreader ebenfalls die Intonation wechselt (Pfad: Überprüfen > Sprache > Sprache für die Korrekturhilfen festlegen). Legen Sie für Worte, Wortgruppen, Absätze oder Abschnitte, die von der Dokumentsprache abweichen (wie beispielsweise englische Fachbegriffe in einem deutschen Text), die entsprechende Sprache fest, damit der Screenreader ebenfalls die Intonation wechselt (Pfad: Überprüfen > Sprache > Sprache für die Korrekturhilfen festlegen).


Dieses Template ist wie folgt gegliedert:
\Cref{sec:demos} zeigt Demonstrationen der LNI-Verlage.
\Cref{sec:lniconformance} zeigt die Einhaltung der Richtlinien durch einfachen Text.

\section{Demonstrationen}
\label{sec:demos}
Das Symbol für Potenzmengen ($\powerset$) wird korrekt angezeigt.
Es ist kein Weierstraß-p ($\wp$) mehr.

Spitze Klammen können direkt eingegeben werden: <test />

Hier eine kleine Demonstration von \href{https://www.ctan.org/pkg/microtype}{microtype}:
\blindtext

\section{Demonstration der Einhaltung der Richtlinien}
\label{sec:lniconformance}

\subsection{Literaturverzeichnis}
Der letzte Abschnitt zeigt ein beispielhaftes Literaturverzeichnis für Bücher mit einem Autor \cite{Ez10} und zwei AutorInnen \cite{AB00}, einem Beitrag in Proceedings mit drei AutorInnen \cite{ABC01}, einem Beitrag in einem LNI Band mit mehr als drei AutorInnen \cite{Az09}, zwei Bücher mit den jeweils selben vier AutorInnen im selben Erscheinungsjahr \cite{Wa14} und \cite{Wa14b}, ein Journal \cite{Gl06}, eine Website \cite{GI19} bzw.\ anderweitige Literatur ohne konkrete AutorInnenschaft \cite{XX14}.
Es wird biblatex verwendet, da es UTF8 sauber unterstützt und \href{https://github.com/gi-ev/LNI/issues/5}{im Gegensatz zu lni.bst} keine Fehler beim bibtexen auftreten.

\newpage

Referenzen sollten nicht direkt als Subjekt eingebunden werden, sondern immer nur durch Authorenanganben:
Beispiel: \Citet{AB00} geben ein Beispiel, aber auch \citet{Az09}.
Hinweis: Großes C bei \texttt{Citet}, wenn es am Satzanfang steht. Dies ist analog zu \texttt{Cref}.

\begin{itemize}
\item	Für Aufzählungen verwenden Sie bitte die Formatvorlage <Aufzählung Ebene 1>.
\item	Aufzählungen haben dieselbe Schriftart und Schriftgröße wie der Fließtext und sind im Blocksatz ausgerichtet. Mehrzeilige Aufzählungspunkte beginnen an derselben Stelle wie die erste Zeile. Für Aufzählungen verwenden Sie bitte die Formatvorlage <Aufzählung Ebene 1 + Block>. Wenn nicht diese Formatvorlage übernommen wird: Einstellung Absatz, Sondereinzug Hängend um 0,88 cm und eine neue Tabstoppposition bei 0,88 cm einfügen und eventuelle andere löschen.
\item	Sie werden in der ersten Ebene mit einem • (Mittenpunkt) begonnen. Es ist ein Abstand von 6 pt nach dem Text einzustellen. 
\begin{itemize}
\item	Für Aufzählungen der zweiten Ebene verwenden Sie bitte die Formatvorlage <Aufzählung Ebene 2>.
\item	In der zweiten Ebene werden Aufzählungen mit einem – (Spiegelstrich) begonnen. Es ist ein Abstand von 6 pt nach dem Text einzustellen. Wenn nicht diese Formatvorlage übernommen wird: Einstellung Absatz, links 0,88 cm, Sondereinzug Hängend um 0,88 cm und eine neue Tabstoppposition bei 1,76 cm einfügen und eventuelle andere löschen.
\item	Aufzählungen sollen generell nicht tiefer als zwei Ebenen sein.
\end{itemize}
\end{itemize}
Sollten nummerierte Aufzählungen verwendet werden, gelten die gleichen Formatierungsvorgaben wie bei unnummerierten: 
\begin{enumerate}
\item	Für nummerierte Aufzählungen verwenden Sie bitte die Formatvorlage <Aufzählung nummeriert Ebene 1>.
\item		Nummerierte Aufzählungen haben die gleiche Formatierung wie unnummerierte Aufzählungen
\item		Die Ausrichtung ist links, der Abstand vom Rand ist 0 cm, der Tabstopp ist bei 0,88 cm zu setzten und der Einzug ist ebenfalls auf 0,88 cm einzustellen. 
\begin{enumerate}
\item		Für nummerierte Aufzählungen der zweiten Ebene verwenden Sie bitte die Formatvorlage <Aufzählung nummeriert Ebene 2>.
\item		Die Formatierung der zweiten Ebene ist äquivalent zu der zweiten Ebene nicht nummerierter Aufzählungen. 
\item		Auch nummerierte Aufzählungen sollen generell nicht tiefer als zwei Ebenen sein.
\end{enumerate}
\end{enumerate}

Die Schriftenreihe LNI umfasst Tagungsbände, die Seminarreihe der Informatiktage und die im Rahmen des Dissertationspreises der GI prämierten Dissertationen. Soweit hier Unterschiede in der Formatierung bestehen, wird im Folgenden explizit darauf hingewiesen. Legen Sie für Worte, Wortgruppen, Absätze oder Abschnitte, die von der Dokumentsprache abweichen (wie beispielsweise englische Fachbegriffe in einem 

Formatierung und Abkürzungen werden für die Referenzen \texttt{book}, \texttt{inbook}, \texttt{proceedings}, \texttt{inproceedings}, \texttt{article}, \texttt{online} und \texttt{misc} automatisch vorgenommen.
Mögliche Felder für Referenzen können der Beispieldatei \texttt{lni-paper-example-de.bib} entnommen werden.
Andere Referenzen sowie Felder müssen allenfalls nachträglich angepasst werden.

\subsection{Abbildungen}
\Cref{fig:demo} zeigt eine Abbildung.

\begin{figure}
  \centering
  \includegraphics[width=.8\textwidth]{example-image}
  \caption{Demographik}
  \label{fig:demo}
\end{figure}

\subsection{Tabellen}
\Cref{tab:demo} zeigt eine Tabelle.

\begin{table}
\centering
\begin{tabular}{lll}
\toprule
Überschriftsebenen & Beispiel & Schriftgröße und -art \\
\midrule
Titel (linksbündig) & Der Titel \ldots & 14 pt, Fett\\
Überschrift 1 & 1 Einleitung & 12 pt, Fett\\
Überschrift 2 & 2.1 Titel & 10 pt, Fett\\
\bottomrule
\end{tabular}
\caption{Die Überschriftsarten}
\label{tab:demo}
\end{table}

\subsection{Programmcode}
Die LNI-Formatvorlage verlangt die Einrückung von Listings vom linken Rand.
In der \texttt{lni}-Dokumentenklasse ist dies für die \texttt{verbatim}-Umgebung realisiert.

\begin{verbatim}
public class Hello {
    public static void main (String[] args) {
        System.out.println("Hello World!");
    }
}
\end{verbatim}

Alternativ kann auch die \texttt{lstlisting}-Umgebung verwendet werden.

\Cref{L1} zeigt uns ein Beispiel, das mit Hilfe der \texttt{lstlisting}-Umgebung realisiert ist.

\begin{lstlisting}[caption={Beschreibung}, label=L1, language=Java]
public class Hello {
    public static void main (String[] args) {
        System.out.println("Hello World!");
    }
}
\end{lstlisting}

\subsection{Formeln und Gleichungen}


%% \bibliography{lni-paper-example-de.tex} ist hier nicht erlaubt: biblatex erwartet dies bei der Preambel
%% Starten Sie "biber paper", um eine Biliographie zu erzeugen.
\printbibliography

\end{document}
